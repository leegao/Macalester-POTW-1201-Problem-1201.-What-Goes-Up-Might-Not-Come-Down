\subsection*{Values}

The problem prompt asks for $P^1_1 = \frac{\bar\gamma}{\sqrt{5}}\left(1 + \frac{1+\bar\gamma^2}{1 - \bar\gamma^2}\right) = \frac{1}{\sqrt{5}}\frac{2}{\sqrt{5}} = \boxed{\frac{2}{5}}$, the probability of succeeding if we start at $(1,1)$.

Similarly, we can also compute $P_0^1 = \frac13 + \frac23 \frac25 = \boxed{\frac35}$ the probability of succeeding if we start at $(1,0)$. We can verify these values with some monte-carlos testing. For example, we can simulate a run using the following python code:

\begin{verbatim}
def simulate(x):
    if not x: return 1
    y = random.randint(-1,1)
    if not y: return 0
    return simulate(x+y)

def simulate1(x):
    y = random.randint(-1,1)
    if not y: return simulate(x)
    return simulate1(x+y)
\end{verbatim}

Doing a simple trial gives me statistics that converges to the values we've computed.